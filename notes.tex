\documentclass[11pt]{article}
\usepackage{geometry, amsmath}
\geometry{letterpaper, left=0.5in, right=0.5in, top=1.0in, bottom=1.0in}
\setlength\parindent{0pt}

\newcommand{\bm}[1]{\mathbf{#1}}
\DeclareMathOperator{\sign}{sgn}
\DeclareMathOperator{\adj}{adj}
\DeclareMathOperator{\rot}{rot}
\DeclareMathOperator{\reflect}{reflect}
\DeclareMathOperator{\invol}{invol}
\DeclareMathOperator{\scale}{scale}
\newcommand{\norm}[1]{\left\|#1\right\|}
\title{Notes from\\\textit{Foundations of Game Engine Development\\Volume 1: Mathematics}}
\author{Matt McKenna}

\begin{document}
\begin{titlepage}
\clearpage\maketitle
\thispagestyle{empty}
\end{titlepage}
\section*{1 Vectors and Matrices}
\subsection*{1.2 Basic Vector Operations}

"It's important to realize that a vector by itself does not have any specific location in space.
The information it posseses is merely an oriented magnitude and nothing more."

\subsubsection*{1.4.2 Matrix Multiplication}
\begin{itemize}
    \item When an $n \times p$ matrix $\bm{A}$ is multiplied by $p \times m$ matrix $\bm{B}$ the
        $(i, j)$ entry of the matrix product $\bm{AB}$ is given by:
        \begin{equation}
            (\bm{AB})_{ij} = \sum_{k=0}^{p-1} A_{ik} B_{kj}
        \end{equation}

    \item The $(i, j)$ entry of $(\bm{AB})^T$ is the $(j, i)$ entry of $\bm{AB}$:
        \begin{equation}
            (\bm{AB})^\text{T}_{ij} = (\bm{AB})_{ji} = \sum_{k=0}^{p-1} A_{jk} B_{ki}
        \end{equation}

    \item Multiplying a $3 \times 3$ matrix $\bm{M}$ multiplied by a $3 \times 1$ column vector
        $\bm{v}$:
        $$
        \bm{Mv} =
        \begin{bmatrix}
            M_{00} & M_{01} & M_{02} \\
            M_{10} & M_{11} & M_{12} \\
            M_{20} & M_{21} & M_{22}
        \end{bmatrix}
        \begin{bmatrix}
            v_x \\
            v_y \\
            v_z
        \end{bmatrix} =
        \begin{bmatrix}
            M_{00} v_x &+& M_{01} v_y &+& M_{02} v_z \\
            M_{10} v_x &+& M_{11} v_y &+& M_{12} v_z \\
            M_{20} v_x &+& M_{21} v_y &+& M_{22} v_z
        \end{bmatrix}
        $$

\item If we write $\bm{M} = [\bm{a}, \bm{b}, \bm{c}]$, then we have:
    \begin{equation}
        \mathbf{Mv} = v_x \mathbf{a} + v_x \mathbf{b} + v_x \mathbf{c}
    \end{equation}
\item Multiplication of transposed matrices:
    \begin{equation}
        (\bm{AB})^T = \bm{B}^T \bm{A}^T
    \end{equation}
\end{itemize}

\subsubsection*{1.5.1 Dot Product}
\begin{equation}
    \bm{a} \cdot \bm{b} = \sum_{i=0}^{n-1} a_i b_i
    = \bm{a}^\text{T} \bm{b}
    = \left\|\bm{a}\right\| \norm{\bm{b}} \cos \theta
\end{equation}
\begin{itemize}
    \item Provides a computationally cheap way to determine how much two vectors are parallel or perpendicular to each other.
    \item If $\bm{a}$ and $\bm{b}$ are unit vectors, then $\bm{a} \cdot \bm{b} = \cos \theta$ and the range of the cosine function is $[-1, 1]$
    \item Assuming $\norm{\bm{a}} = \norm{\bm{b}}$, $\bm{a} \cdot \bm{b}$ is:
    \begin{itemize}
        \item Maximally positive when $\bm{a}$ and $\bm{b}$ point in the same direction.
        \item Maximally negative when $\bm{a}$ and $\bm{b}$ point in opposite directions.
        \item Zero when $\bm{a}$ and $\bm{b}$ are perpendicular, regardless of magnitude
    \end{itemize}
    \item Squared magnitude of a vector:
        \begin{equation}
            v^2 = \bm{v} \cdot \bm{v} = \norm{\bm{v}}^2
        \end{equation}
\end{itemize}

\subsubsection*{1.5.2 Cross Product}
\begin{equation}
    \bm{a} \times \bm{b} =
        \left(a_y b_z - a_z b_y, a_z b_x - a_x b_z, a_x b_y - a_y b_x\right)
\end{equation}

\begin{itemize}
\item Can also be expressed as a matrix product by forming a special $3 \times 3$ antisymmetric matrix denoted by
    \begin{equation}
        \left[ \bm{a} \right]_\times =
        \begin{bmatrix}
            0 & -a_z & a_y \\
            a_z & 0 & -a_x \\
            -a_y & a_x & 0
        \end{bmatrix}
    \end{equation}
    \begin{equation*}
        \bm{a} \times \bm{b} = \left[\bm{a}\right]_\times \bm{b} =
        \begin{bmatrix}
            0 & -a_z & a_y \\
            a_z & 0 & -a_x \\
            -a_y & a_x & 0
        \end{bmatrix}
        \begin{bmatrix}
            b_x \\
            b_y \\
            b_z
        \end{bmatrix} =
        \begin{bmatrix}
            -a_z b_y + a_y b_z \\
            a_z b_x - a_x b_z \\
            -a_y b_x + a_x b_y
        \end{bmatrix}
    \end{equation*}
\item Only defined for 3-dimensions, whereas dot product is defined for all numbers of dimensions
\item Actually a subtle misinterpretation of a more general and more algebraically sound operation called the \textit{wedge product}.
\item Zero when $\bm{a}$ and $\bm{b}$ are parallel
\item When $\bm{a}$ and $\bm{b}$ are not parallel, $\bm{a} \times \bm{b}$ is a new vector that is perpendicular to both $\bm{a}$ and $\bm{b}$
\item Has a magnitude equal to the area of the parallelogram having sides $\bm{a}$ and $\bm{b}$:
    \begin{equation}
        \norm{\bm{a} \times \bm{b}} = \norm{\bm{a}} \norm{\bm{b}} \sin \theta
    \end{equation}
\item Other identities:
    \begin{align*}
        \bm{a} \times \bm{b} &= -\bm{b} \times \bm{a} \\
        \bm{a} \times \left( \bm{b} \times \bm{c} \right) &=
            \bm{b} \left(\bm{a} \cdot \bm{c} \right)
            - \bm{c} \left(\bm{a} \cdot \bm{b} \right)
    \end{align*}
\end{itemize}

\subsubsection*{1.5.3 Scalar Triple Product}
\begin{equation} \label{eqn:triple}
    \bm{\left[ a, b, c \right]} =
    \left(\bm{a} \times \bm{b}\right) \cdot \bm{c} =
    \left(\bm{b} \times \bm{c}\right) \cdot \bm{a} =
    \left(\bm{c} \times \bm{a}\right) \cdot \bm{b}
\end{equation}

\subsection*{1.6 Vector Projection}
\begin{itemize}
\item Projection of $\bm{a}$ onto $\bm{b}$:
    \begin{equation}
        \bm{a}_{\left\|\bm{b}\right.} =
        \frac{\bm{a} \cdot \bm{b}}{b^2} \bm{b}
    \end{equation}

\item $\bm{a}_{\left||b\right.}$ indicates the component of the vector $\bm{a}$ that is parallel to the vector $\bm{b}$ (alternatively $\text{proj}_\bm{b} \bm{a}$)
    \begin{equation}
        \bm{a}_{\left\|\bm{b}\right.} =
        \frac{1}{b^2} \bm{b}\bm{b}^\text{T} \bm{a} =
        \frac{1}{b^2}
        \begin{bmatrix}
        b_x^2 & b_x b_y & b_x b_z \\
        b_x b_y & b_y^2 & b_y b_z \\
        b_x b_z & b_y b_z & b_z^2
        \end{bmatrix}
    \end{equation}

\item Where $\bm{b} \bm{b}^\text{T}$ is a symmetric matrix and an example of an \textit{outer product}
    \begin{equation}
        \bm{u} \otimes \bm{v} = \bm{u} \bm{v}^\text{T} =
        \begin{bmatrix}
        u_x \\
        u_y \\
        u_z
        \end{bmatrix}
        \begin{bmatrix}
        v_x & v_y & v_z
        \end{bmatrix} =
        \begin{bmatrix}
        u_x v_x & u_x v_y & u_x v_z \\
        u_y v_x & u_y v_y & u_y v_z \\
        u_z v_x & u_z v_y & u_z v_z
        \end{bmatrix}
    \end{equation}
\item If we subtract the projection $\bm{a}_{\left||b\right.}$ from the original vector $\bm{a}$, the we get the part that is perpendicular to the vector $\bm{b}$, called the rejection of $\bm{a}$ from $\bm{b}$.
    \begin{equation}
        \bm{a}_{\perp \bm{b}} = \bm{a} - \bm{a}_{\left\|\bm{b}\right.} = \bm{a} - \frac{\bm{a} \cdot \bm{b}}{b^2} \bm{b}
    \end{equation}
\item From a set of $n$ linearly independent vectors $\left\{ \bm{v}_1, \bm{v}_2, \ldots, \bm{v}_n \right\}$, the \textit{Gram-Schmidt process} can be used to produce a set of mutually orthogonal vectors $\left\{  \bm{u}_1, \bm{u}_2, \ldots, \bm{u}_n \right\}$.  For example, a set of 3 vectors $\left\{  \bm{v}_1, \bm{v}_2, \bm{v}_3 \right\}$ is orthogonalized using the calculations:
    $$
    \begin{aligned}
        \bm{u}_1 &= \bm{v}_1 \\
        \bm{u}_2 &=
            \bm{v}_2
            - \left( \bm{v}_2 \right)_{\left|| \bm{u}_1 \right.} \\
        \bm{u}_3 &=
            \bm{v}_3
            - \left( \bm{v}_3 \right)_{\left|| \bm{u}_1 \right.}
            - \left( \bm{v}_3 \right)_{\left|| \bm{u}_2 \right.}\\
    \end{aligned}
    $$

    It is common that the vectors $\bm{u}_i$ be renormalized to unit length after the orthogonalization process
\end{itemize}

\subsection*{1.7 Matrix Inversion}
The identity matrix:
$$
\bm{I}_n =
\begin{bmatrix}
    1 & 0 & \ldots & 0 \\
    0 & 1 & \ldots & 0 \\
    \vdots & \vdots & \ddots & \vdots \\
    0 & 0 & \ldots & 1 \\
\end{bmatrix}
$$
which allows us to define the inverse of a matrix:
\begin{equation} \bm{M}^{-1} \bm{M} = \bm{M}\bm{M}^{-1} = \bm{I} \end{equation}

\subsubsection*{1.7.2 Determinants}
\begin{itemize}
\item A matrix has an inverse if, and only if, its determinant is not zero
\item The determinant of any $n \times n$ matrix $\bm{M}$ can be expressed using the Leibniz formula for determinants:
    \begin{equation}
        \det(\bm{M}) = \sum_{\sigma \in S_n} \left( \sign \left(\sigma\right) \prod_{k=0}^{n-1} M_{k,\sigma(k)} \right)
        \label{eqn:leibniz}
    \end{equation}
\item Using expansion by minors, , the determinant of an $n \times n$
    matrix $\bm{M}$ is given by:
    \begin{equation} \label{eqn:minors}
        \det(\bm{M}) = \sum_{j=0}^{n-1} M_{kj} (-1)^{k+j} \left|\bm{M}_{\overline{kj}}\right|
    \end{equation}
    where $\bm{M}_{\overline{ij}}$ is the submatrix that excludes row $i$ and
    column $j$, and $k$ can be chosen to be any fixed row in the matrix.
\end{itemize}

\subsubsection*{1.7.3 Elementary Matrices}
There are 3 elementary row operations:
\begin{enumerate}
    \item Multiply one row of matrix $\bm{M}$ by a nonzero scalar value $t$
        \begin{equation}
            \bm{E} = \begin{bmatrix}
                1 & \ldots & 0 & \ldots & 0 \\
                \vdots & \ddots & \vdots & & \vdots \\
                0 & \ldots & t & \ldots & 0 \\
                \vdots & & \vdots & \ddots & \vdots \\
                0 & \ldots & 0 & \ldots & 1
            \end{bmatrix}
            \leftarrow \text{row } r
        \end{equation}
        This causes the determinant of $\bm{M}$ to be multiplied by $t$.
    \item Exchange 2 rows of $\bm{M}$
        \begin{equation}
            \bm{E} = \begin{bmatrix}
                1 & \ldots & 0 & \ldots & 0 & \ldots & 0\\
                \vdots & \ddots & \vdots & & \vdots & & \vdots \\
                0 & \ldots & 0 & \ldots & 1 & \ldots & 0 \\
                \vdots & & \vdots & \ddots & \vdots & & \vdots \\
                0 & \ldots & 1 & \ldots & 0 & \ldots & 0 \\
                \vdots & & \vdots & & \vdots & \ddots & \vdots \\
                0 & \ldots & 0 & \ldots & 0 & \ldots & 1
            \end{bmatrix}
            \begin{matrix}
                \phantom{0}\\
                \phantom{\vdots}\\
                \leftarrow \text{row } r\\
                \phantom{\vdots}\\
                \leftarrow \text{row } s \\
                \phantom{\vdots}\\
                \phantom{0}\\
            \end{matrix}
        \end{equation}
        This causes the determinant of $\bm{M}$ to be negated.  Because of this,
        the determinant of $\bm{M}$ is zero if any 2 rows are the same.
    \item Add a scalar multiple of one row of $\bm{M}$ to another row of $\bm{M}$
        \begin{equation}
        \begin{aligned}
            &\phantom{=} \begin{matrix}
                \phantom{0} & \phantom{\ldots} & \phantom{0} & \phantom{ll} & \text{column } s & \phantom{\ldots} & \phantom{0}\\
                \phantom{0} & \phantom{\ldots} & \phantom{0} & \phantom{ll} & \downarrow & \phantom{\ldots} & \phantom{0}\\
            \end{matrix} \\
            \bm{E} &= \begin{bmatrix}
                1 & \ldots & 0 & \ldots & 0 & \ldots & 0\\
                \vdots & \ddots & \vdots & & \vdots & & \vdots \\
                0 & \ldots & 0 & \ldots & 1 & \ldots & 0 \\
                \vdots & & \vdots & \ddots & \vdots & & \vdots \\
                0 & \ldots & 1 & \ldots & 0 & \ldots & 0 \\
                \vdots & & \vdots & & \vdots & \ddots & \vdots \\
                0 & \ldots & 0 & \ldots & 0 & \ldots & 1
            \end{bmatrix}
            \begin{matrix}
                \phantom{0}\\
                \phantom{\vdots}\\
                \leftarrow \text{row } r\\
                \phantom{\vdots}\\
                \leftarrow \text{row } s \\
                \phantom{\vdots}\\
                \phantom{0}\\
            \end{matrix}
        \end{aligned}
        \end{equation}
        This does not change the determinant of $\bm{M}$.
\end{enumerate}

\subsubsection*{1.7.4 Inverse Calculation}
If it exists, the inverse of a matrix can be found using \textit{Gauss-Jordan elimination},
where elementary row operations are successively applied to the matrix until it
is transformed into the identity matrix.
\begin{itemize}
    \item General method for square matrices of any size
    \item For matrices of smaller size, faster methods exist
\end{itemize}

    \subsubsection*{1.7.5 Inverses of Small Matrices}
    This approach uses the minors with alternating signs that appear in the formula
    for the determinant given by \eqref{eqn:minors}.
    \begin{itemize}
    \item The \textit{cofactor} of the $(i, j)$ entry of $\bm{M}$:
        \begin{equation}
            C_{ij} \left(\bm{M}\right) = \left(-1\right)^{i+j} \left| \bm{M}_{\overline{ij}} \right|
        \end{equation}
    \item The \textit{cofactor matrix} $\bm{C}\left(\bm{M}\right)$ of an $n \times n$
        matrix $\bm{M}$ is the matrix in which every entry of $\bm{M}$ has been
        replaced by the corresponding cofactor.
    \item The formula for the inverse of a matrix $\bm{M}$ using its cofactor matrix:
        \begin{equation}
            \bm{M}^{-1} = \frac{\bm{C}^{\text{T}} \left(\bm{M}\right)}{\det \left(\bm{M}\right)}
        \end{equation}
    \item The matrix $\bm{C}^{\text{T}}$ is called the \textit{adjugate} of the matrix $\bm{M}$,
        and it is denoted by $\adj \left(\bm{M}\right)$
    \item For a $2 \times 2$ matrix $\bm{A}$, the explicit inverse formula is
        \begin{equation}
            \bm{A}^{-1} = \frac{1}{A_{00}A_{11} - A_{01}A_{10}}
            \begin{bmatrix}
                A_{11} & -A_{01} \\
                -A_{10} & -A_{00}
            \end{bmatrix}
        \end{equation}
    \item For a $3 \times 3$ matrix $\bm{B}$, the explicit inverse formula is
        \begin{equation}
            \bm{B}^{-1} = \frac{1}{\det \left(\bm{B}\right)}
            \begin{bmatrix}
                B_{11}B_{22}-B_{12}B_{21} & B_{02}B_{21}-B_{01}B_{22} & B_{01}B_{12}-B_{02}B_{11} \\
                B_{12}B_{20}-B_{10}B_{22} & B_{00}B_{22}-B_{02}B_{20} & B_{02}B_{10}-B_{00}B_{12} \\
                B_{10}B_{21}-B_{11}B_{20} & B_{01}B_{20}-B_{00}B_{21} & B_{00}B_{11}-B_{01}B_{10}
            \end{bmatrix}
        \end{equation}
        Note that each row in this formula is a cross product of two columns of the
        matrix $\bm{B}$, and the determinant is equal to the triple product \eqref{eqn:triple} of the
        three columns of the matrix.
    \item The inverse of a matrix $\bm{M} = \bm{ \left[ a, b, c \right] }$
        whose column are the 3D vectors $\bm{a}$, $\bm{b}$, and $\bm{c}$ can be written as
        \begin{equation}
            \bm{M}^{-1} = \frac{1}{\left[ \bm{a}, \bm{b}, \bm{c} \right]}
            \begin{bmatrix}
                \bm{b} \times \bm{c} \\
                \bm{c} \times \bm{a} \\
                \bm{a} \times \bm{b} \\
            \end{bmatrix}
        \end{equation}
        where the cross products are treated as row vectors.
    \item Let $\bm{M}$ be a $4 \times 4$ matrix whose first 3 rows are filled by the
        four 3D column vectors $\bm{a}$,$\bm{b}$,$\bm{c}$, and $\bm{d}$, and whose
        fourth row contains the entries $\begin{bmatrix} x & y & z & w \end{bmatrix}$
        \begin{equation}
            \bm{M} =
            \begin{bmatrix}
                \uparrow & \uparrow & \uparrow & \uparrow \\
                \bm{a} & \bm{b} & \bm{c} & \bm{d} \\
                \downarrow & \downarrow & \downarrow & \downarrow \\
                - & - & - & - \\
                x & y & z & w
            \end{bmatrix}
        \end{equation}
        Then we define the four vectors $\bm{s}$, $\bm{t}$, $\bm{u}$, and $\bm{v}$, as
        \begin{align}
            \bm{s} &= \bm{a} \times \bm{b} \\
            \bm{t} &= \bm{c} \times \bm{d} \\
            \bm{u} &= y\bm{a} - x\bm{b} \\
            \bm{v} &= w\bm{c} - z\bm{d}
        \end{align}
        The determinant takes the form
        \begin{equation}
            \det \left(\bm{M}\right) = \bm{s} \cdot \bm{v} + \bm{t} \cdot \bm{u}
        \end{equation}
        and the inverse of $\bm{M}$ is given by
        \begin{equation}
            \bm{M}^{-1} = \frac{1}{\bm{s} \cdot \bm{v} + \bm{t} \cdot \bm{u}}
            \left[
            \begin{array}{c | r}
                \bm{b} \times \bm{v} + y\bm{t} & -\bm{b} \cdot \bm{t} \\
                \bm{v} \times \bm{a} - x\bm{t} & \bm{a} \cdot \bm{t} \\
                \bm{d} \times \bm{u} + w\bm{s} & -\bm{d} \cdot \bm{s} \\
                \bm{u} \times \bm{c} + z\bm{s} & \bm{c} \cdot \bm{s}
            \end{array}
            \right]
    \end{equation}
\end{itemize}

\section*{2 Transformations}
\subsection*{2.1 Coordinate Spaces}
\subsubsection*{2.1.1 Transformation Matrices}
\begin{itemize}
    \item The transformationfrom a position $\bm{p}_A$ in coordinate system $A$
        to the position $\bm{p}_B$ in coordinate system $B$ can be expressed as
        \begin{equation}
            \bm{p}_B = \bm{M}\bm{p}_A + \bm{t}
        \end{equation}
        where $\bm{M}$ is a $3 \times 3$ matrix that reorients the coordinate axes,
        and $\bm{t}$ is a 3D translation vector that moves the origin of the
        coordinate system.
\end{itemize}
\subsubsection*{2.1.2 Orthogonal Transforms}
    \begin{itemize}
        \item The inverse of an orthogonal matrix is equal to its transpose.
            Assuming that $\bm{a}$, $\bm{b}$, and $\bm{c}$ all have unit
            length and are mutually perpendicular
            \begin{equation}
                \bm{M}^\text{T}\bm{M} =
                \begin{bmatrix}
                    \leftarrow & \bm{a}^\text{T} & \rightarrow \\
                    \leftarrow & \bm{b}^\text{T} & \rightarrow \\
                    \leftarrow & \bm{c}^\text{T} & \rightarrow \\
                \end{bmatrix}
                \begin{bmatrix}
                    \uparrow & \uparrow & \uparrow \\
                    \bm{a} & \bm{b} & \bm{c} \\
                    \downarrow & \downarrow & \downarrow
                \end{bmatrix} =
                \begin{bmatrix}
                    a^2 & \bm{a} \cdot \bm{b} & \bm{a} \cdot \bm{c} \\
                    \bm{b} \cdot \bm{a} & b^2 & \bm{b} \cdot \bm{c} \\
                    \bm{c} \cdot \bm{a} & \bm{c} \cdot \bm{b} & c^2
                \end{bmatrix}=
                \begin{bmatrix}
                    1 & 0 & 0 \\
                    0 & 1 & 0 \\
                    0 & 0 & 1
                \end{bmatrix}
            \end{equation}
            Since $\bm{a}$, $\bm{b}$, and $\bm{c}$ each have unit length
            and are perpendicular, all diagonal entries are ones and
            all other entries are zero.  This means $\bm{M}^\text{T}\bm{M} = \bm{I}$
            and, therefore, $\bm{M}^\text{T} = \bm{M}^{-1}$.
        \item Orthogonal matrices preserve the dot product between any two vectors
            $\bm{a}$ and $\bm{b}$.  Given the vectors after they are
            transformed by orthogonal matrix $\bm{M}$
            \begin{equation}
                \left(\bm{M}\bm{a}\right) \cdot \left(\bm{M}\bm{b}\right) =
                \left(\bm{M}\bm{a}\right)^\text{T} \cdot \left(\bm{M}\bm{b}\right) =
                \bm{a}^\text{T}\bm{M}^\text{T} \cdot \bm{M}\bm{b} =
                \bm{a}^\text{T} \bm{b} = \bm{a} \cdot \bm{b}
                \label{eqn:ortho_dot_proof}
            \end{equation}
            Since $\bm{a} \cdot \bm{a}$ is the squared magnitude of $\bm{a}$,
            \eqref{eqn:ortho_dot_proof} also proves that magnitude is not
            changed by an orthogonal matrix.  It must therefore also be
            true that the angle $\theta$ between $\bm{a}$ and $\bm{b}$
            is unchanged.
        \item The transform performed by an orthogonal matrix is always
            a rotation, a reflection, or a combination of the two.
        \item The determinant of an orthogonal matrix is always $\pm 1$,
            positive for a pure rotation and negative for a rotation
            with a reflection.
    \end{itemize}

\subsubsection*{2.1.3 Transform Composition}
\begin{itemize}
    \item Whenever a vector $\bm{v}$ is transformed by a matrix $\bm{M}_1$, then
        by a matrix $\bm{M}_2$, the result $\bm{v}'$ is calculated by:
        \begin{equation}
            \bm{v}' = \bm{M}_2 \left( \bm{M}_2 \bm{v} \right)
            = \left( \bm{M}_2 \bm{M}_1 \right) \bm{v}
        \end{equation}
    \item To perform a transform $\bm{A}$ in coordinate system $A$ in coordinate
        system $B$, where matrix $\bm{M}$ transforms vectors from $A$ to $B$,
        the equivalient transform $\bm{B}$ in coordinate system $B$ is
        \begin{equation}
            \bm{B} = \bm{M} \bm{A} \bm{M}^{-1}
        \end{equation}
\end{itemize}

\subsection*{2.2 Rotations}
\subsubsection*{2.2.1 Rotation About a Coordinate Axis}
\begin{align}
    \bm{M}_{\rot x} \left(\theta\right) &= \begin{bmatrix}
        1 & 0 & 0 \\
        0 & \cos \theta & -\sin \theta \\
        0 & \sin \theta & \cos \theta
    \end{bmatrix} \\
    \bm{M}_{\rot x} \left(\theta\right) &= \begin{bmatrix}
        \cos \theta & 0 & \sin \theta \\
        0 & 1 & 0 \\
        -\sin \theta & 0 & \cos \theta
    \end{bmatrix} \\
    \bm{M}_{\rot x} \left(\theta\right) &= \begin{bmatrix}
        \cos \theta & -\sin \theta & 0\\
        \sin \theta & \cos \theta & 0\\
        0 & 0 & 1
    \end{bmatrix}
\end{align}

\subsubsection*{2.2.2 Rotation About an Arbitrary Axis}
For a vector $\bm{v}$ rotated about a vector$\bm{a}$ by an angle $\theta$ where the angle
between $\bm{v}$ and $\bm{a}$ is $\alpha$
\begin{equation}
    \bm{v}' = \bm{v}_{\parallel \bm{a}} + \bm{v}_{\perp \bm{a}} \cos \theta
        + \left( \bm{a} \times \bm{v} \right) \sin \theta
\end{equation}
which is expressed in matrix format as
\begin{equation}
    \bm{M}_{\rot} \left(\theta, \bm{a}\right) = \begin{bmatrix}
        \cos\theta + (1 - \cos\theta) a_x^2 & (1 - \cos\theta) a_xa_y - (\sin\theta) a_z & (1 - \cos\theta) a_xa_z + (\sin\theta)a_y \\
        (1 - \cos\theta) a_xa_y + (\sin\theta) a_z & \cos\theta + (1 - \cos\theta) a_y^2 & (1 - \cos\theta) a_ya_z - (\sin\theta)a_x \\
        (1 - \cos\theta) a_xa_z - (\sin\theta) a_y & (1 - \cos\theta) a_ya_z + (\sin\theta) a_x & \cos\theta + (1 - \cos\theta) a_z^2
    \end{bmatrix}
\end{equation}

\subsubsection*{2.3 Reflections}
Vector $\bm{v}$ can be reflected through a plane perpendicular to vector $\bm{a}$ (assuming $\bm{a}$ has unit length)
\begin{equation}
    \bm{v}' = \bm{v}_{\perp \bm{a}} - \bm{v}_{\parallel \bm{a}}
\end{equation}
with matrix representation
\begin{equation}
    \bm{v}' = \begin{bmatrix}
        1 - a_x^2 & -a_xa_y & -a_xa_z \\
        -a_xa_y & 1 - a_y^2 & -a_ya_z \\
        -a_xa_z & -a_ya_z & 1-a_z^2
    \end{bmatrix} \bm{v} - \begin{bmatrix}
        a_x^2 & a_xa_y & a_xa_z \\
        a_xa_y & a_y^2 & a_ya_z \\
        a_xa_z & a_ya_z & a_z^2
    \end{bmatrix}
\end{equation}
Combining the matrix terms into a single matrix, we arrive at the formula
\begin{equation}
    \bm{M}_{\reflect}\left(\bm{a}\right) = \begin{bmatrix}
        1 - 2a_x^2 & -2a_xa_y & -2a_xa_z \\
        -2a_xa_y & 1 - 2a_y^2 & -2a_ya_z \\
        -2a_xa_z & -2a_ya_z & 1 - 2a_z^2 \\
    \end{bmatrix}
\end{equation}
We can also construct a transform that negates the the perpendicular component instead of the
parallel component
\begin{equation}
    \bm{v}' = \bm{v}_{\parallel \bm{a}} - \bm{v}_{\perp \bm{a}}
\end{equation}
with the matrix form
\begin{equation}
    \bm{M}_{\invol}\left(\bm{a}\right) = \begin{bmatrix}
        2a_x^2 - 1 & 2a_xa_y & 2a_xa_z \\
        2a_xa_y & 2a_y^2 - 1 & 2a_ya_z \\
        2a_xa_z & 2a_ya_z & 2a_z^2 - 1 \\
    \end{bmatrix}
\end{equation}
where $\bm{M}_{\invol} \left(\bm{a}\right)$ denotes $\bm{M}$ as an \textit{involution}, which is a
matrix that, when multiplied by itself, produces the identity matrix.

\subsection*{2.4 Scales}
A scale transformation aligned to the coordinate axes
\begin{equation}
    \bm{M}_{\scale} \left(s_x, s_y, s_z\right) = \begin{bmatrix}
        s_x & 0 & 0 \\
        0 & s_y & 0 \\
        0 & 0 & s_z
    \end{bmatrix}
\end{equation}
To scale a vector $\bm{v}$ along a single arbitrary direction $\bm{a}$ while preserving the vector's
size in every direction orthogonal to $\bm{a}$
\begin{equation}
    \bm{v}' = s\bm{v}_{\parallel \bm{a}} + \bm{v}_{\perp \bm{a}}
\end{equation}
where the transformation matrix is
\begin{equation}
    \bm{M}_{\scale} \left(s, \bm{a}\right) = \begin{bmatrix}
        (s-1)a_x^2 + 1 & (s-1)a_xa_y & (s-1)a_xa_z \\
        (s-1)a_xa_y & (s-1)a_y^2 + 1 & (s-1)a_ya_z \\
        (s-1)a_xa_z & (s-1)a_ya_z & (s-1)a_z^2 + 1
    \end{bmatrix}
\end{equation}
\end{document}
